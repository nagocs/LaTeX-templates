%-------------------------------------------------------------------
% Fájl: 00_preambulum.tex
%-------------------------------------------------------------------

% Packagek
%% Dokumentum konfiguráció
\usepackage[english]{babel} % dokumentum nyelve
\usepackage{t1enc} % dokumentum kódolása
\usepackage{geometry} % dokumentum formátum
\usepackage{hyperref} % hivatkozások
\usepackage[backend=biber,style=numeric,sorting=none]{biblatex} % irodalomjegyzék

%% Ábrák, grafikák
\usepackage{graphicx}
\usepackage{float}
\usepackage{caption}
\usepackage{subcaption}
\usepackage{tikz}
\usepackage{pdfpages}

%% Matematikai és fizikai írásmód
\usepackage{amsmath,amsthm,amsfonts,amssymb,amscd} % minden matek
\usepackage{esint} % integrálok
\usepackage{nicefrac}
\usepackage{mathtools}
\usepackage{siunitx}

%% Fejléc
\usepackage{fancyhdr}

%% Tagolás
\usepackage{titlesec}
\usepackage{csquotes}

%% Táblázatok
\usepackage{booktabs}
\usepackage{multirow}

%% Szövegkiemelések
\usepackage[breakable,listings,skins]{tcolorbox} % színes keret
\usepackage{framed} % keret

%% Dokumentum tagolás
\usepackage{subfiles}

%-------------------------------------------------------------------
% Dokumentum beállítások
%% Dokumentum formátum
\geometry{a5paper, total={130mm,175mm}, left=10mm, top=15mm}
\setlength\parindent{0pt}
\pagestyle{empty}

%% Fejléc beállítása
\pagestyle{fancyplain}
\headheight 35pt
\renewcommand{\headrulewidth}{0pt}
\rhead{\leftmark}
\chead{}
\lhead{\empty}
\lfoot{}
\cfoot{\small\thepage}
\rfoot{}
\headsep 1.5em

%% Mértékegységek formátuma
\sisetup{exponent-product = \cdot,per-mode=fraction}

%% Irodalomjegyék beállítása
\addbibresource{contents/literature.bib}

%% Táblázat formátum beállítása
\newcolumntype{L}[1]{>{\raggedright\let\newline\\\arraybackslash\hspace{0pt}}m{#1}}
\newcolumntype{C}[1]{>{\centering\let\newline\\\arraybackslash\hspace{0pt}}m{#1}}
\newcolumntype{R}[1]{>{\raggedleft\let\newline\\\arraybackslash\hspace{0pt}}m{#1}}
\renewcommand{\arraystretch}{1.25}

%% Számozások beállítása fejezetenként
\numberwithin{equation}{section}
\numberwithin{figure}{section}
\numberwithin{table}{section}

%% Fejezet címek beállításai
%\titlelabel{\thetitle.\quad}

%% Felsorolási szintek
\renewcommand\labelitemii{$\to$}
\renewcommand\labelitemiii{-}

%% Ábrák útvonala
\graphicspath{{../figures/}}
