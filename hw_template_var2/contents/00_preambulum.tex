% -------------------------------------------------------------------
% Fájl: 00_preambulum.tex
% -------------------------------------------------------------------

% Packagek
%% Dokumentum konfiguráció
\usepackage[english]{babel} % dokumentum nyelve
\usepackage{t1enc} % dokumentum kódolása
\usepackage{geometry} % dokumentum formátum
\usepackage{hyperref} % hivatkozások
%\usepackage[backend=biber,style=numeric,sorting=none]{biblatex} % irodalomjegyzék

%% Ábrák, grafikák
\usepackage{graphicx}
\usepackage{float}
\usepackage{caption}
\usepackage{tikz}
\usepackage{pdfpages}
\usepackage{structmech}

%% Matematikai és fizikai írásmód
\usepackage{amsmath,amsthm,amsfonts,amssymb,amscd} % matek
\usepackage{nicefrac}
\usepackage{mathtools}
\usepackage{siunitx}

%% Fejléc, lábléc
\usepackage{fancyhdr}
\usepackage{lastpage} % utolsó oldal sorszáma

%% Dokumentumtagolás
\usepackage{titlesec}
\usepackage{csquotes}
\usepackage{booktabs}

%% Animáció
%\usepackage{animate}

% -------------------------------------------------------------------
% Dokumentum beállítások
%% Dokumentum formátum
\geometry{a4paper,total={170mm,240mm},left=20mm,top=30mm}
\setlength\parindent{0pt}

%% Fejléc, lábléc beállítása
\pagestyle{fancy}
\renewcommand{\headrulewidth}{0.4pt}
\renewcommand{\footrulewidth}{0.4pt}
\headheight 1.5em
\rhead{\NEV{}, \NEPTUN{}}
\chead{}
\lhead{\TANTARGYKOD{}, \TANTARGY{}}
\lfoot{}
\cfoot{}
\rfoot{\small\thepage\ / \pageref{LastPage}}
\headsep 1.5em


%% Mértékegységek formátuma
\sisetup{exponent-product = \cdot, per-mode=fraction}

%% Tikz
\usetikzlibrary{math}
\usetikzlibrary{calc}
\usetikzlibrary{angles}
\usetikzlibrary{quotes}

%% Színek
\definecolor{YELLOW}{HTML}{f2f1c1}
\definecolor{GREEN}{HTML}{2d5c38}
\definecolor{RED}{HTML}{a23635}
\definecolor{BLUE}{HTML}{2a3373}
\definecolor{BROWN}{HTML}{62482d}
\definecolor{GRAY}{HTML}{e5e5e0}

%% Irodalomjegyék beállítása
%\addbibresource{contents/literature.bib}

%% Táblázat formátum beállítása
\newcolumntype{L}[1]{>{\raggedright\let\newline\\\arraybackslash\hspace{0pt}}m{#1}}
\newcolumntype{C}[1]{>{\centering\let\newline\\\arraybackslash\hspace{0pt}}m{#1}}
\newcolumntype{R}[1]{>{\raggedleft\let\newline\\\arraybackslash\hspace{0pt}}m{#1}}
\renewcommand{\arraystretch}{1.25}

%% Számozások beállítása fejezetenként
% \numberwithin{equation}{section}
% \numberwithin{figure}{section}
% \numberwithin{table}{section}

%% Fejezet címek beállításai
% \titlelabel{\thetitle.\quad}

%% Felsorolási szintek
\renewcommand\labelitemii{$\to$}
\renewcommand\labelitemiii{-}

%-------------------------------------------------------------------
% Parancsok
%% Matematikai parancsok
\renewcommand{\d}[1]{\mathrm{d}#1}
\renewcommand{\vec}[1]{\mathbf{#1}}

%% Egyéb parancsok
\newcommand{\specialcell}[2][c]{\begin{tabular}[#1]{c}#2\end{tabular}}
\newcommand*\circled[1]{\begin{tikzpicture}[baseline=(C.base)] \node[draw,circle,inner sep=2pt](C) {#1}; \end{tikzpicture}}


