%-------------------------------------------------------------------
% Fájl: 00_commands.tex
%-------------------------------------------------------------------

% Parancsok
%% Szövegtagolás
\newenvironment{step}[2]{\vspace{3pt} #1 \textit{#2.}}{}
\newcommand{\quantity}[3]{\begin{equation*} #1 \quad \textrm{$\left[\:#2\:\right]$} \end{equation*}}

%% Matematikai környezetek
\theoremstyle{plain}
\newtheorem{theorem}{Theorem}[section]
\theoremstyle{definition}
\newtheorem{definition}{Definition}[section]
\newtheorem*{example}{Example}
\newtheorem*{remark}{Remark}

%% Kiemelés
\newtcolorbox{alertblock}[1][]{
  colback=red!5!white,
  colframe=red!70!black,
  fonttitle=\bfseries,
  title={#1},
  sharp corners
}
\newtcolorbox{block}[1][]{
  colback=blue!30!black!10!white,
  colframe=blue!30!black,
  fonttitle=\bfseries,
  title={#1},
  sharp corners,
  before skip = 6pt,
  after skip = 12pt,
  enhanced jigsaw,
  breakable
}
\newtcolorbox[auto counter,number within=section]{exampleblock}[1]{
  coltitle=black,
  colback=green!30!black!10!white,
  colframe=green!30!black,
  fonttitle=\bfseries,
  title={Example~\thetcbcounter: #1},
  attach title to upper,
  after title={\newline},
  sharp corners
}

%% Szimbólumok
%%% Alapok
\newcommand{\jo}{j\omega}
\newcommand{\dd}[1]{\:\mathrm{d}#1}
\renewcommand{\vec}[1]{\mathbf{#1}}
\newcommand{\cross}{\times}

%%% Transzformációk
\newcommand{\Tr}[2]{\mathbf{#1}\left\{#2\right\}}
\newcommand{\invTr}[2]{\mathbf{#1}^{-1}\left\{#2\right\}}

%%% Komplex számok
\renewcommand{\Re}[1]{\textrm{Re}\left\{#1\right\}}
\renewcommand{\Im}[1]{\textrm{Im}\left\{#1\right\}}

%%% Matematikai operátorok
\DeclareMathOperator{\grad}{\mathbf{grad}}
\let\div\relax
\DeclareMathOperator{\div}{\mathbf{div}}
\DeclareMathOperator{\rot}{\mathbf{rot}}
\DeclareMathOperator{\laplacian}{\Delta}
\DeclareMathOperator{\del}{\nabla}
\DeclareMathOperator{\tr}{\mathbf{tr}}
\DeclareMathOperator{\dev}{dev}
\DeclareMathOperator{\sph}{sph}

%%% Kontinuummechanika specifikus
\newcommand{\motion}[1]{\boldsymbol\chi\left(#1\right)}
\newcommand{\diag}[1]{\mathbf{diag}\left(#1\right)}
\newcommand{\invmotion}[1]{\boldsymbol\chi^{-1}\left(#1\right)}
\DeclareMathOperator{\Grad}{\mathbf{Grad}}
\DeclareMathOperator{\Div}{\mathbf{Div}}
\DeclareMathOperator{\Rot}{\mathbf{Rot}}
\newcommand{\obj}[2][\empty]{\mathring{#2}^{#1}}

%% Egyenlet levezetés
\newcommand{\eqline}{\noalign{\smallskip} \hline \noalign{\smallskip}}
\newcommand{\eqcomment}[1]{& \triangleright~ & #1}
\newenvironment{eqwhere}[2]
{\par\vspace{-0.5\abovedisplayskip}\begin{tabular}{C{0.25cm}||C{1.5cm}L{#1}L{#2}}}
{\end{tabular}\par\vspace{\belowdisplayskip}}

%% Egyéb
\newcommand*\circled[1]{\begin{tikzpicture}[baseline=(C.base)] \node[draw,circle,inner sep=1pt](C) {#1}; \end{tikzpicture}}
